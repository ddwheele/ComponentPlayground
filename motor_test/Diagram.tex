\documentclass{article}

\usepackage[siunitx]{circuitikz}
\usepackage{geometry}
\geometry{letterpaper, landscape, margin=1cm}

\begin{document}

\begin{circuitikz}
	[pin/.style={rectangle, draw, inner sep=0pt, minimum height=1cm, minimum width=0.5cm}]

	% Draw a grid
	%\clip(0,-10) rectangle (15,10);
	\draw[step=1cm,gray,ultra thin] (0,-10) grid (20,10);
	\draw (0,0) circle [radius=10pt];
		
	% Body of the ESP32
	\node at (8,7.5) [rectangle, thick, draw, minimum width=16cm, minimum height=4cm]  {};
	\node at (8,7.5) [font=\bf]{ESP32};
	
	% ESP32 Pins
	\node (3V3) at (1,6) [pin] {} ;
	\node at (1,7) {3V3};
	\node (GNDESP) at (2,6) [pin]{} +(1,.5);
	\node at (2,7) {GND};
	\node (D4) at (5,6) [pin] {} +(1,.5);	
	\node at (5,7) {D4};
	\node (D5) at (9,6) [pin] {} +(1,.5);
	\node at (9,7) {D5};
	\node (D18) at (10,6) [pin] {} +(1,.5);
	\node at (10,7) {D18};
	\node (D19) at (11,6) [pin] {} +(1,.5);
	\node at (11,7) {D19};
	\node (VP) at (14,9) [pin] {} +(1,.5);
	\node at (14,8) {VP};

	% Body of the TB6612
	\node at (10.5,0) [rectangle, thick, draw, minimum width=11cm, minimum height=5cm]  {};
	\node at (10,0) [font=\bf]{TB6612};
			
	% TB6612 Pins
	\node (PWMA) at (6,2) [pin] {} +(1,.5);
	\node at (6,0.5) {PWMA};
	\node (AIN2) at (7,2) [pin] {} +(1,.5);
	\node at (7,1) {AIN2};
	\node (AIN1) at (8,2) [pin] {} +(1,.5);
	\node at (8,1) {AIN1};
	\node (STBY) at (9,2) [pin] {} +(1,.5);
	\node at (9,1) {STBY};	
	\node (BIN1) at (10,2) [pin] {} +(1,.5);	
	\node at (10,1) {BIN1};
	\node (BIN2) at (11,2) [pin] {} +(1,.5);	
	\node at (11,1) {BIN2};
	\node (PWMB) at (12,2) [pin] {} +(1,.5);
	\node at (12,0.5) {PWMB};
	\node (GNDTB_TOP) at (13,2) [pin] {} +(1,.5);
	\node at (13,1) {GND};
	\node (VCC) at (14,2) [pin] {} +(1,.5);	
	\node at (14,1) {VCC};
	\node (VM) at (15,2) [pin] {} +(1,.5);
	\node at (15,1) {VM};
	
	\node (MOTORA1) at (8,-2) [pin] {} +(1,.5);	
	\node (MOTORA2) at (9,-2) [pin] {} +(1,.5);	
	\node at (8.5,-1) {MOTORA};
	\node (GNDTB_BOT1) at (10,-2) [pin] {} +(1,.5);		
	\node (GNDTB_BOT2) at (11,-2) [pin] {} +(1,.5);	
	\node at (10.5,-1) {GND};
	\node (MOTORB1) at (12,-2) [pin] {} +(1,.5);	
	\node (MOTORB2) at (13,-2) [pin] {} +(1,.5);	
	\node at (12.5,-1) {MOTORB};
	
	% Motors - Gives compile error when I try to make motors regular nodes
	\draw (8,-4) node[elmech](motor) {M};
	\draw (MOTORA1.south) to (motor.north);
	\draw (motor.south)  -- ++(0,-0.5) --  ++(1,0) -- (MOTORA2.south);
	
	%\draw (12,-5) node[elmech](motor) {Right};
	%\draw (MOTORB1.south) to (motor.north);
	%\draw (motor.south)  -- ++(0,-1) --  ++(1,0) -- (MOTORB2.south);
	
	% ESP connections
	\draw (GNDESP.south) -- ++(0,-10) -- ++ (0,-1) node(GROUND)[ground]{};

	% TB6612 connections
	\draw (D4.south)  to[short] (AIN1.north);
	\draw (D5.south)  to[short] (AIN2.north);	
	\draw (D18.south)  to[short] (BIN1.north);
	\draw (D19.south)  to[short] (BIN2.north);
	
%	vsourceAMshape

	\node (5V) at (18,5) [battery] {5V} ;
	
%	\draw (VM.north) -- ++(0,1) -- ++(3,0)   to[battery, v=$5V$] ++(0,2) ;

	
% draw the button circuits
%	\draw (VIN.east)  to[short,-*] ++(1,0)
%		to[normally open push button, draw, color=red] ++(0,-6)
%		to[short]  +(0, -1)
%		to[short]  (D25.east);  
%	 \draw (3.5,0)  to[R=10<\kilo \ohm>,*-*] +(9,0); 
	  
%	\draw (VIN.east)  to[short] ++(2,0)
%		to[normally open push button, draw, color=green] ++(0,-6)
%		to[short]  +(0, -3)
%		to[short]  (D32.east);  
%	 \draw (4.5,-2)  to[R=10<\kilo \ohm>,*-*] +(8,0); 

% draw the LED circuits
%	\draw (D26.east)  to[leDo, color=red] ++(8,0) 
%		   to [R=\SI{220}{\ohm}]   ++(1,0)
%		   	  to[short, -*]  ++(1,0);
		 
%	\draw (D33.east)  to[leDo, color=green] ++(8,0) 
%		   to [R=\SI{220}{\ohm}]   ++(1,0)
%		  to[short, -*]  ++(1,0);

\end{circuitikz}

\end{document}