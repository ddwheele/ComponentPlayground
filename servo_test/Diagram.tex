%%%%%%%%%%%%%%%%%%%%%%%%%%
%
% servo_test diagram
%
%%%%%%%%%%%%%%%%%%%%%%%%%%
% to convert a .pdf to .svg, open in Inkscape and export as .svg

% Goes with the Knob servo test

\documentclass{article}

\usepackage[siunitx]{circuitikz}
\usepackage{geometry}
\geometry{letterpaper, landscape, margin=1cm}

\begin{document}

\begin{circuitikz}
	[pin/.style={rectangle, draw, inner sep=0pt, minimum height=1cm, minimum width=0.5cm}]

	% Draw a grid
	%\clip(0,-10) rectangle (15,10);
	%\draw[step=1cm,gray,ultra thin] (0,-10) grid (20,10);
	%\draw (0,0) circle [radius=10pt];
		
	% Body of the Feather
	\node at (8,7.5) [rectangle, thick, draw, minimum width=16cm, minimum height=4cm]  {};
	\node at (8,7.5) [font=\bf]{Bluefruit 32u4};
	
	% ESP32 Pins
	\node (3V3) at (1,6) [pin] {} ;
	\node at (1,7) {3V3};
	\node (GNDESP) at (2,6) [pin]{} +(1,.5);
	\node at (2,7) {GND};
	\node (f9) at (5,6) [pin] {} +(1,.5);	
	\node at (5,7) {9~};
	%\node (D5) at (9,6) [pin] {} +(1,.5);
	%\node at (9,7) {D5};
	\node (A0) at (10,6) [pin] {} +(1,.5);
	\node at (10,7) {A0};
	%\node (D19) at (11,6) [pin] {} +(1,.5);
	%\node at (11,7) {D19};
	%\node (VP) at (14,9) [pin] {} +(1,.5);
	%\node at (14,8) {VP};

	% Body of the TB6612
	\node at (10.5,0) [rectangle, thick, draw, minimum width=11cm, minimum height=5cm]  {};
	\node at (10,0) [font=\bf]{TB6612};
			
	% TB6612 Pins
	\node (PWMA) at (6,2) [pin] {} +(1,.5);
	\node at (6,0.5) {PWMA};
	\node (AIN2) at (7,2) [pin] {} +(1,.5);
	\node at (7,1) {AIN2};
	\node (AIN1) at (8,2) [pin] {} +(1,.5);
	\node at (8,1) {AIN1};
	\node (STBY) at (9,2) [pin] {} +(1,.5);
	\node at (9,1) {STBY};	
	%\node (BIN1) at (10,2) [pin] {} +(1,.5);	
	%\node at (10,1) {BIN1};
	%\node (BIN2) at (11,2) [pin] {} +(1,.5);	
	%\node at (11,1) {BIN2};
	%\node (PWMB) at (12,2) [pin] {} +(1,.5);
	%\node at (12,0.5) {PWMB};
	\node (GNDTB_TOP) at (13,2) [pin] {} +(1,.5);
	\node at (13,1) {GND};
	\node (VCC) at (14,2) [pin] {} +(1,.5);	
	\node at (14,1) {VCC};
	\node (VM) at (15,2) [pin] {} +(1,.5);
	\node at (15,1) {VM};
	
	\node (MOTORA1) at (8,-2) [pin] {} +(1,.5);	
	\node (MOTORA2) at (9,-2) [pin] {} +(1,.5);	
	\node at (8.5,-1) {MOTORA};
	\node (GNDTB_BOT1) at (10,-2) [pin] {} +(1,.5);		
	\node (GNDTB_BOT2) at (11,-2) [pin] {} +(1,.5);	
	\node at (10.5,-1) {GND};
	\node (MOTORB1) at (12,-2) [pin] {} +(1,.5);	
	\node (MOTORB2) at (13,-2) [pin] {} +(1,.5);	
	\node at (12.5,-1) {MOTORB};
	
	% Servo
	\tikzset{flipflop myJK/.style={flipflop,
 		flipflop def={t1=GND, t2=P, t3=S}}
	}
	\draw (13,-6) node[flipflop myJK](servo) {Srv};
	\draw (MOTORA1.south)  -- ++(0,-4.5)  to (servo.pin 3);
	\draw (VM.north) -- ++(0,1) -- ++(2,0) -- ++(0,-7)  -- ++(-8,0) -- ++(0,-2.5) to (servo.pin 2);

	% Grounds
	\draw (GNDESP.south) -- ++(0,-0.5) -- ++ (0,0.5) node(GROUND)[ground]{};
	\draw (servo.pin 1)  -- ++(-0.5,0) node(GROUND)[ground]{}; 
	\draw  (AIN2.north) -- ++ (0,1.5) -- ++ (-1,0) node(GROUND)[ground]{};
	
	% TB6612 connections
	\draw (f9.south)  -- ++(0,-2.5) --  ++(1,0) to[short] (PWMA.north);

	
	% 3V3 Power connections
	\draw[red] (3V3.south)  -- ++(0,-1) --  ++(7,0) -- (AIN1.north);
	\draw[red] (3V3.south)  -- ++(0,-1) --  ++(8,0) -- (STBY.north);
	
	% 5V Power connection
	\draw(15,2.5) -- ++(0,1.5)  to[battery, v=$5V$] ++(-2,0)  -- ++(0,-1.5) ;

	% Resistor
	\draw [red] (3V3.south) -- ++(0,-1) -- ++(8,0) 
		to[pR,-, color=black] [black] ++(2,0); 
	
	\draw(A0.south) -- ++(0,-1) ;
	\draw (11,4.5)  -- ++(0,-0.5) node(GROUND)[ground]{}; 

\end{circuitikz}

\end{document}
















